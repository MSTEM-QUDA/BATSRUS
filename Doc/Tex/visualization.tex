%  Copyright (C) 2002 Regents of the University of Michigan, 
%  portions used with permission 
%  For more information, see http://csem.engin.umich.edu/tools/swmf
\section{Tecplot}

Tecplot is a visualization package created by Amtec Engineering, Inc.
out of Bellevue, Washington.
The package was originally designed for visualization of the output of
computational fluid dynamics (CFD) codes and now is a good multipurpose
visualization package.  
The strengths of Tecplot are the point and click interface, the 
wide range of options and the ability to produce high quality
three dimensional images in postscript, encapsulated postscript, tiff
and other formats.
The software comes with detailed documentation for processing data and creating
images.
Starting in 2017, Tecplot provides the PyTecplot (at https://www.tecplot.com/docs/pytecplot/)
package for connecting Python scripts to the Tecplot 360 visualization engine.

Currently BATS-R-US supports output in Tecplot ASCII format, which can be
processed by preplot into \verb|.plt| binary format.
There is also an option of \verb|.dat| output of ASCII header with binary data
and connectivity, which is more efficient but cannot be processed by preplot.
This file can be read and converted into VTK files by the Julia package mentioned in Section \ref{sec:julia}.
The generated unstructured grid \verb|.vtu| files can be read by Tecplot.
The direct support for Tecplot binary formats \verb|.plt| and \verb|.szplt|
may be included in the future.

Tecplot requires a license.

\section{IDL}

IDL is well suited to visualize 1D and 2D data on structured and unstructured
(AMR) grids, and 3D data on structured grids. 
It is a full programming language that can perform complex data processing and
visualization tasks efficiently.
The IDL programs developed for the SWMF can be found in the {\tt share/IDL/}
directory including detailed documentation.

IDL requires a license, although it can be used in demo mode for free.
In demo mode each session is limited to 7 minutes, which is actually 
sufficient to create plots.

\section{MATLAB}

MATLAB is a multi-paradigm numerical computing environment and proprietary
programming language developed by MathWorks.
Link to the MATLAB package VisAnaMatlab is can be found in the {\tt share/MATLAB/} directory.
This package can read, process, and plot SWMF data, similarly to the IDL scripts.

MATLAB requires a license.

\section{Julia} \label(sec:julia)

Julia is a high-level, high-performance, dynamic programming language
well-suited for high-performance numerical analysis and computational sciences.
The VisAnaJulia package, available in {\tt share/Julia}, is developed to read,
process and visualize SWMF output files.
It also provides the functionality of converting Tecplot \verb|.dat| format
into VTK format.
Documentations can be found inside the package.

Julia is open source, as is the VisAnaJulia package.

\section{VisIt}

VisIt is a high performance visualization package built upon the VTK library.
It can read the HDF5 output with extension .batl.

Visit is free.

\section{ParaView}

ParaView is another high performance visualization package built upon the VTK
library.
The Tecplot \verb|.dat| files from BATS-R-US can be converted into VTK files
with the VisAnaJulia package for processing in ParaView.

Paraview is free.

\section{Python}

Python is a very popular object oriented programming language. 
SpacePy has been developed to read and visualize SWMF output. 
The swmfpy package in {\tt share/Python} 
can perform various tasks related to the SWMF. 

Python, SpacePy, swmfpy are all open source.

