%^CFG COPYRIGHT UM
\chapter{Introduction \label{chapter:introduction}}

The \BATSRUS\ code is a first principles magnetohydrodynamic (MHD)
model which has been used to simulate the Earth's magnetosphere, the
Heliosphere, the magnetosphere of most of the planets and various comets.  
The code can be extended for use to any problem for which the MHD equations
are a reasonable physical model.

The \BATSRUS\ code is the most important building block of the
Space Weather Modeling Framework (SWMF). The SWMF executes and
couples a number physics models, componenst as a single model.
The \BATSRUS\ is used in a triple role in SWMF: 
it can model the Solar Corona (SC component), 
the Inner Heliosphere (IH component) and the 
the Global Magnetoshere (GM component).
A lot of effort was spent on making sure that the very same 
source code, scripts, test suites and makefiles are used in
the stand alone \BATSRUS\ and in the various components.
We tried to change the behavior of the standalone version
as little as possible. 

\BATSRUS\ stands for Block Adaptive Tree Solar-wind Roe Upwind Scheme.
This name, while not complete in describing the code, especially in its
newer incarnations, points out some of \BATSRUS' main features.  Specifically,
\BATSRUS\ originally solved the MHD equations using a finite volume upwind 
Roe-type scheme.
Currently there are several different solvers available.  The computation
region in \BATSRUS\ is made up of logically Cartesian blocks of cells that can be 
adaptively refined to give higher resolution in a restricted part of the 
domain.  The division of blocks into smaller blocks creates a tree like
structure of blocks, where a divided block has eight children, and the blocks
are connected to other blocks much like the branches of a tree.
Finally, \BATSRUS\ is most commonly run  to model the solar wind
interaction with solar system bodies.  

This document is aimed at providing the user detailed information about
installation, compilaton and execution of the code, and 
how one can change the physical and numerical parameters
to achieve the desired result. The tools provided for visualization
are also described.

The physics and the numerics contained in the code are described in
the DESIGN document. That dociment should help
the user understand the design philosophy behind the code, the 
available physics that the code contains and the numerical algorithms that
make the code work. 

\section{Acknowledgments}

\BATSRUS\ was developed at the University of Michigan starting in 1996
with funding under the NASA High Performance Computing and Communications (HPCC)
Earth and Space Sciences (ESS) program (NASA ESS Cooperative Agreement 
Number: NCCS5-146).  Continued work is funded by NFS (implicit) and KDI.
The project's principle investigator is Tamas Gombosi.

Contributions to the development of \BATSRUS\ fall roughly into three
categories: theoretical development, code development and scientific
investigations.   The principle players and their involvement is as follows:

\begin{tabbing}
{\bf Principle Investigators and Theory} \\
Tamas Gombosi \hspace{.25in} \= 1994- \hspace{0.35in} \= 
                          Principle Investigator, MHD theory \\
Ken Powell     \> 1994-     \> Co-Principle Investigator, numerics and algorithm development \\
Quentin Stout  \> 1994-     \> Co-Principle Investigator, 
                          parallel architecture and computer science \\
Darren DeZeeuw \> 1996-     \> Co-Investigator, Principle code developer \\
\> \> \\
{\bf \BATSRUS\ Code Development} \\
Darren DeZeeuw \> 1996-     \> Co-Investigator, Principle \BATSRUS\ code developer, \\
               \>           \> Principle IMM developer  \\
Hal Marshall   \> 1996-1998 \> Code development, computer science \\
Clinton Groth  \> 1997-1999 \> Code development, CFD \\ 
Gabor Toth     \> 1999-     \> Algorithm and code development \\
Igor Sokolov   \> 2001-     \> Algorithm and code development \\
\> \> \\
{\bf Code Development and Science} \\
Aaron Ridley   \> 1999-     \> Ionospheric science and \BATSRUS\ code development,\\
               \>           \> Principle UAM developer \\
Bob Oehmke     \> 1997-     \> Code development and computer science \\
K.C. Hansen    \> 1999-     \> Magnetospheric science and code development \\
Chip Manchester\> 2000-     \> Heliospheric science and code development \\
Ilia Roussev   \> 2001-     \> Heliospheric science and code development
\end{tabbing}

\section{Web Pages for \BATSRUS}
Information about \BATSRUS\ and the Center for Space Environment Modeling
can be found at
\begin{verbatim}
http://csem.engin.umich.edu/
\end{verbatim}
