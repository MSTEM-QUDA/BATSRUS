%^CFG COPYRIGHT UM
\chapter{The Synoptic Solar Wind Model \label{chapter:synopticSW}}

\section{General Description of the Model \label{section:SWmodel}}
The synoptic solar wind model is designed to provide the the ambient physical 
conditions for the Solar Corona (SC), the Inner Heliosphere (IH), and the 
Solar Wind (SW). 
The model is called ``synoptic'' due to the fact that it is driven by synoptic 
maps of the observed surface radial magnetic field of the Sun throughout a 
whole Carrington Rotation (one rotation period of the Sun in approximately 27 
days). These maps are used to provide the inner boundary conditions for the 
steady-state MHD solution in the domain between the Sun and the Earth.

There are two main issues when one tries to create numerical model for the 
solar corona. First, one needs to specify the initial condition for the 
three-dimensional configuration of the magnetic field. Since the only 
constrain on the magnetic field is the observed radial field on the surface, 
we use the common ``Potential Field'' approximation to specify the initial 
field in the whole domain (see Section \ref{section:potentialfield}). The 
other issue is the solar wind heating and accelerating mechanism. This issue 
is discussed in Section \ref{section:semiempirical}.      

%%%%%%%%%%%%%%%%%%%%%%%%%%%%%%%%%%%%%%%%%%%%%%%%%%%%%%%%%%%%%%%%%%%%%%%%%%%%%%%
% The potential field approximation                                           %
%%%%%%%%%%%%%%%%%%%%%%%%%%%%%%%%%%%%%%%%%%%%%%%%%%%%%%%%%%%%%%%%%%%%%%%%%%%%%%%

\section{The potential field approximation \label{section:potentialfield}}

The solar corona is dominated by its magnetic field. Therefore, it is 
important to know what is the three dimensional structure of the magnetic 
field in order to study the physical processes in the corona. Since the solar 
magnetic field can be routinely measured only at the photosphere, where the 
plasma density is high enough for measuring the Zeeman Splitting, one needs 
to find a way to approximate the global structure of the coronal magnetic 
field. The most commonly used method to approximate the coronal magnetic 
field is the so-called 'potential field' approximation (Altschuler and 
Newkirk, Solar Physics, 9:131-149, 1969). In this approximation, it is 
assumed that the Alfv\'en speed is much larger than the speed of convective 
motions in the low corona, so the field line relaxation time is much shorter 
than the typical advection time scale. In other words, the field line will 
respond quickly to any motion we apply on it (this motion can be seen as 
electric current) so in practice the magnetic field is static. Under the 
assumption that there are no currents in a physical system, we can write 
Ampere's law as follows:
\begin{equation}
\label{eq:Amper1}
\nabla \times \mathbf{B}=0,
\end{equation}
and we can write $\mathbf{B}$ as a gradient of some scalar potential $\psi$:
\begin{equation}
\label{eq:grad}
\mathbf{B}=-\nabla \psi.
\end{equation}
Since we also know that 
\begin{equation}
\label{eq:DivB}
\nabla \cdot \mathbf{B}=0,
\end{equation}
combining eq. \ref{eq:grad} with eq. \ref{eq:DivB} results in the Laplace equation 
for the scalar potential:
\begin{equation}
\label{eq:Laplace}
\nabla^2\psi=0.
\end{equation}
The general solution for this equation in spherical coordinates for \\
$R_0 < r < R_s$ is:
\begin{eqnarray}
\label{eq:GaneralSol}
\psi(r,\theta,\phi)=\mathop{\sum}_{n=1}^{\infty}\mathop{\sum}_{m=0}^{n}
\left[ R_0\left( \frac{R_0}{r}\right)^{n+1}+R_s \cdot c_n \left( \frac{r}{R_s}
\right)^{n} 
\right] \nonumber \\
\times \left(g_n^m \cos{m\phi} + h_n^m \sin{m\phi} \right)P_{nm}(\theta),
\end{eqnarray}
which gives $\psi=0$ at $r=R_s$ for the choice of $c_n=-\left( \frac{R_0}{R_s} 
\right)^{n+2}$, 
particularly as 
$R_s \rightarrow \infty$. 
$P_{nm}$ are the associated Legendre Polynomials, which are a function of 
$\cos{\theta}$. 
The coefficients 
$g_n^m$ and $h_n^m$ can be determined from the photospheric magnetic field 
data and have magnetic field dimension.

Following the above solution, we can obtain the solution for the magnetic 
field components (Altschuler et al. 1969, eqs. 8-10):

\begin{eqnarray}
\label{eq:Br2}
B_r=-\frac{\partial \psi}{\partial r}=\mathop{\sum}_{n=1}^{\infty} 
\mathop{\sum}_{m=0}^{n} 
\left[ (n+1)\left( \frac{R_0}{r} \right)^{n+2}-n\left( \frac{r}{R_s} \right)
^{n-1}c_n \right] 
\nonumber \\
\times \left(g_n^m \cos{m\phi} + h_n^m \sin{m\phi} \right)P_n^m(\theta),
\end{eqnarray}

\begin{eqnarray}
\label{eq:Btheta2}
B_\theta=-\frac{1}{r}\frac{\partial \psi}{\partial \theta}=-\mathop{\sum}_{n=1}
^{\infty} 
\mathop{\sum}_{m=0}^{n} 
\left[ \left( \frac{R_0}{r} \right)^{n+2}+c_n \left( \frac{r}{R_s} \right)
^{n-1} \right] 
\nonumber \\
\times \left(g_n^m \cos{m\phi} + h_n^m \sin{m\phi} \right)\frac{\partial 
P_n^m(\theta)}{\partial \theta},
\end{eqnarray}

\begin{eqnarray}
\label{eq:Bphi2}
B_\phi=-\frac{1}{r\sin{\theta}}\frac{\partial \psi}{\partial \phi}=
\mathop{\sum}_{n=1}^{\infty} \mathop{\sum}_{m=0}^{n} 
\left[\left( \frac{R_0}{r} \right)^{n+2}+c_n \left( \frac{r}{R_s} \right)
^{n-1} \right]\nonumber \\
\times \frac{m}{\sin{\theta}} \left(g_n^m \sin{m\phi} - h_n^m \cos{m\phi} 
\right)P_n^m(\theta).
\end{eqnarray}

We can determine the harmonic coefficients from the observed photospheric 
values of $B_r$ as follows. The orthogonality relation over a sphere with 
$r=1$ for the general Legendre functions is:
\begin{eqnarray}
\label{eq:SphNorm}
\frac{1}{4\pi}\mathop{\int}_{\theta=0}^{\pi}\mathop{\int}_{\phi=0}^{2\pi}
P_{nm}(\theta)  \left\{ \begin{array}{l l}  \cos{m\phi}\\  \sin{m\phi}\\ 
\end{array} 
\right\}
P_{n'm'}(\theta) \left\{ \begin{array}{l l}  \cos{m'\phi}\\  \sin{m'\phi}\\ 
\end{array} 
\right\}
\sin{\theta} d\theta d\phi= \nonumber \\
W\delta_{nn'}\delta_{mm'},
\end{eqnarray}
where $W$ is the normalization factor.
For the general Legendre functions,
\begin{equation}
\label{eq:GenNorm}
W=\frac{2}{2n+1}\frac{(n+m)!}{(n-m)!},
\end{equation}
and
\begin{equation}
\label{eq:SchmidtNorm}
W=\frac{1}{2n+1}
\end{equation}
for the Schmidt normalization, so the relation between the Schmidt ($P_n^m$) 
and the general Legendre functions ($P_{nm}$) is:
\begin{equation}
\label{eq:NormRel}
P_n^m(\theta)=\left\{ 2 \frac{(n-m)!}{(n+m)!} \right\} ^{1/2}P_{nm}(\theta).
\end{equation}
In \BATSRUS, the polynomials are calculated with 
the Schmidt normalization.  
For $r=R_0=1$, the radial magnetic field becomes:
\begin{equation}
\label{eq:BrSum}
B_r(\theta,\phi)= \mathop{\sum}_{n=1}^{\infty}\mathop{\sum}_{m=0}^{n}
R_n\left\{ \begin{array}{l l}  g_n^m\\  h_n^m\\ \end{array} \right\}P_n^m
(\theta)\left\{ \begin{array}{l l}  \cos{m\phi}\\  \sin{m\phi}\\ \end{array} 
\right\},
\end{equation}
where $R_n=\left[ (n+1)+n\left( \frac{1}{R_s} \right)^{2n+1}  \right]$.\\
Following eq. \ref{eq:SphNorm}, we can obtain the harmonic coefficients from the 
photospheric magnetic data, assuming the Schmidt normalization of the Legendre 
functions (Altschuler et al. 1969):
\begin{equation}
\label{eq:Coeff}
\left\{ \begin{array}{l l}  g_n^m\\  h_n^m\\ \end{array} \right\}=
\frac{2n+1}{4\pi R_n}\mathop{\int}_{\theta=0}^{\pi}\mathop{\int}_{\phi=0}
^{2\pi}B_r(r=R_\odot,\theta,\phi) P_n^m(\theta) 
\left\{ \begin{array}{l l}  \cos{m\phi}\\  \sin{m\phi}\\ \end{array} \right\} 
\sin{\theta} d\theta d\phi,
\end{equation}
where
\[B_r = \left\{ 
\begin{array}{l l}
B_{magnetogram} & \quad \mbox{for radial magnetogram,}\\
\frac{B_{magnetogram}}{\sin{\theta}} & \quad \mbox{for Line-of-Sight 
magnetogram.}\\
\end{array} \right\}. \]
The discretized version of eq. 14 is (Altschuler et al. 1969, eq. 15):
\begin{equation}
\label{eq:CoeffDisc}
\left\{ \begin{array}{l l}  g_n^m\\  h_n^m\\ \end{array} \right\}=
\frac{1}{A}\frac{2n+1}{R_n}\mathop{\sum}_{i=0}^{N_\theta-1} \mathop{\sum}_{j=0}
^{N_\phi-1}
B_r(i,j)\cdot da_{i,j}\cdot P_n^m(\theta_i)
\left\{ \begin{array}{l l}  \cos{m\phi_j}\\  \sin{m\phi_j},\\ \end{array} 
\right\}
\end{equation}
where $da_{i,j}=\sin{\theta}_i d\theta d\phi$ and 
$A=\mathop{\sum}_{i=0}^{N_\theta-1} \mathop{\sum}_{j=0}^{N_\phi-1}da_{i,j}=
4\pi$ for $r=R_\odot$. \\

In SWMF, there is a utility tool to calculate the spherical harmonic 
coefficients from raw magnetogram. The tool is located at: 
\begin{verbatim}
SWMF_DIR/util/DATAREAD/srcMagnetogram
\end{verbatim}
This directory contains the following README file with instructions how 
to create the input harmonics file needed for the SC model:
\begin{verbatim}
##########################################################################
# How to create a magnetogram input file for SWMF from a raw magnetogram #
# fits file:                						 #
##########################################################################

These are the steps for creating a magnetogram file for SWMF from 
any raw magnetogram fits file.

1. Install SWMF (Config.pl -install).
2. Compile the HARMONICS executable by typing:
	make HARMONICS
in the directory SWMF_path/util/DATAREAD/srcMagnetogram. This will 
create the HARMONICS.exe executable in the directory SWMF_path/bin

3. For convenient, you can create a link to this executable from the path
SWMF_path/util/DATAREAD/srcMagnetogram by typing:
	ln -s ../../../bin/HARMONICS.exe HARMONICS.exe
4. Type:
	cp your_magnetoram_file.fits fitsfile.fits
5. Convert the fits file to ASCII format by running the idl program 
fits_to_ascii.pro. You will be asked to insert the maximum order of 
harmonics and the Carrington Rotation number. It is recommended (but not 
required ) to use not more than 90 harmonics, since the computation time 
can be very long. 
   The idl routine generates three files:
	*fitsfile.dat - ASCII file to be used by HARMONICS.exe to calculate 
	 the harmonic coefficients.
	*fitsfile.H - the header of the original fits file with information 
	 about the magnetogram source.
	*fitsfile_tec.dat - a Tecplot file to display the original magnetogram.
6. Run HARMONICS.exe. This executable can be run in parallel mode for faster 
computation. This run will generate a file called harmonics.dat that 
can be used in SWMF. For convenient, it is recommended to rename the file with 
the following naming format:
	cp harmonics.dat CRxxxx_OBS.dat
where xxxx is the Carrington Rotation number and OBS is the observatory name 
(MDI,WSO,MWO,GONG etc.)
7. Move the magnetogram harmonics file to the path defined in the PFSSM flag 
in PARAM.in file (usually run/SC).
\end{verbatim}

\begin{bf}\begin{Large}Note: this routine does not interpolate missing data 
or the polar flux. You have to make sure that the raw magnetogram is properly 
processed!!!!\end{Large}\end{bf}


%%%%%%%%%%%%%%%%%%%%%%%%%%%%%%%%%%%%%%%%%%%%%%%%%%%%%%%%%%%%%%%%%%%%%%%%%%%%%%%
% Semi-empirical model for the solar wind                                     %
%%%%%%%%%%%%%%%%%%%%%%%%%%%%%%%%%%%%%%%%%%%%%%%%%%%%%%%%%%%%%%%%%%%%%%%%%%%%%%%

\section{Semi-empirical Model for the Solar Wind \label{section:semiempirical}}

Numerical reproduction of the solar corona steady-state conditions has been 
extensively 
investigated since the famous work by (Pneuman and Kopp, Solar Physics, 18:258, 
1971).Traditionally, the deposition of energy and/or momentum into the solar 
wind has been described by means of some empirical source terms (Usmanov 93, 
McKenzie 97, Mikic 99, Suess 99, Wu 99, Groth 00). In these models, 
the sources of plasma heating and solar wind acceleration are typically 
modeled in a qualitative sense, and the spatial profiles for the deposition 
of the energy or momentum are usually modeled by exponentials in radial 
distance. In more realistic models, the solar wind is heated and accelerated 
by the energy and momentum interchange between the solar plasma and 
large-scale Alfv\'en turbulence (Jacques 77, Dewar 70, Barnes 92, 
Usmanov 00 and 03).
 
Due to the small number of observations at 1~AU, it is reasonable to adopt 
semi-empirical models. Assimilating a long history of solar wind observations, 
these models are very efficient and quite accurate. A particular example is 
the Wang-Sheeley-Arge model (WSA, Arge and Pizzo 00,Arge et. al 04). 
This model uses the observed photospheric magnetic field to determine the 
coronal field configuration, which is then used to estimate the distribution 
of the final speed of the solar wind, $u_{sw}$. The common disadvantage of 
semi-empirical models is that they are physically incomplete. 

We use the model by Cohen et. al (07) to obtain the steady-state SC and IH 
solution. The SC and IH modules of SWMF are versions of the BASTRUS global 
MHD code, which is fully parallel and has adaptive mesh refinement 
capabilities (see Powell 99). Our SC model is driven by high-resolution 
SOHO MDI magnetograms. We use the magnetogram data to calculate the potential 
magnetic field, assuming the source surface is at $R_{ss}=2.5R_\odot$, where 
$R_\odot$ is the solar radius, and use this distribution of the magnetic field 
as an initial condition. 

The heating and acceleration of the solar wind plasma are achieved by using a 
non-uniform spatial distribution of $\gamma$. In order to obtain a more 
realistic distribution, we use the empirical Wang-Sheeley-Arge (WSA) model as 
an input to our model. The WSA model uses the potential field distribution to 
obtain the magnetic flux tube expansion factor defined as 
(Wang and Sheeley 90):

\begin{equation}
\label{expansionfactor}
f_s=\frac{|B(R_{s})|R^2_{s}}{|B(R_0)| R^2_0}.
\end{equation}

The WSA model provides an empirical relation for the spherical distribution 
of the solar wind speed at 1AU as a function of $f_s$ and the angular distance 
of a magnetic field footpoint from the coronal hole boundary, $\theta_b$. 
In our model, we use the following formula (Arge et. al 2004):

\begin{equation} 
\label{eq:WSA1}
u_{sw}  = 265 + \frac{1.5}{(1+f_s)^{1/3}} \left \{ 5.9 - 1.5 e^{\left[1 -
\left(\theta_b / 7 \right)^{5/2} \right]} \right \}^{7/2}~\mbox{km s}^{-1}.
\end{equation}

A more up-to-date formula (after personal communication with N. Arge 2006) is:
\begin{equation}
\label{eq:WSA2}
u_{sw}=240+\frac{675}{(1+f_s)^{1/4.5}}
\{ 1-0.8\frac{e^{[1-(\theta_b/2.8)^{5/4}]}}{e^1}  \}^3\;km\;s^{-1} 
\end{equation}

We should note two important issues about the WSA model. First, one should be 
aware of the fact that the WSA solution depends on the magnetogram resolution 
and the is not the same for different observatories. This is due to the 
different mapping of the potential field and the expansion factor. Second, 
the WSA fitting is done for 1AU, while we use it in the model at the source 
surface. This is the main reason for deviations of the MHD solution from the 
WSA solution. 

In order to relate the surface value of $\gamma$ to the WSA solar wind speeds, 
we assume that far from the Sun the total energy is dominated by the energy of 
the bulk motion and that the thermal and gravitational energy are negligible. 
We also assume that at the coronal base the bulk kinetic energy is zero. 
Due to energy conservation, we can use the Bernoulli equation to relate the 
two ends of a streamline (or magnetic field line):

\begin{equation}
\label{eq:Bernoulli2}
\frac{u_{sw}^2(\theta,\phi)}{2}=\frac{\gamma_0(\theta_0,\phi_0) }
{[\gamma_0(\theta_0,\phi_0)-1]}
\frac{p_0(\theta_0,\phi_0)}{\rho_0(\theta_0,\phi_0)}-\frac{GM_\odot}{R_\odot}.
\end{equation}

Here $u_{sw}$ is the input solar wind speed from the WSA model, $G$ is the 
gravitational constant, and $M_\odot$ is the solar mass. $\gamma_0$, $p_0$, 
and $\rho_0$ are the photospheric values for the polytropic index, pressure, 
and  mass density. The coordinates $\theta_0,\phi_0$ represent the location 
of the field line footpoint, $u_{sw}(\theta,\phi)$ originated from. We 
interpolate $\gamma$ from its photospheric value to a spherically uniform 
value of $1.1$ on the source surface at $r=2.5R_\odot$. $\gamma$ is linearly 
varied from $1.1$ to $1.5$ for $2.5R_\odot < r < 12.5R_\odot$, and 
$\gamma=1.5$ above $12.5R_\odot$.Once the spatial distribution of $\gamma$ 
is obtained, we solve the MHD equations self-consistently using this location 
dependent polytropic index in the energy equation to obtain the steady state 
solution for the SC and solar wind.

The above distribution of $\gamma$ enables us to reproduce the bi-modal 
structure of the solar wind speed. However, the distributions of the coronal 
density and temperature are still not determined. It is known that the faster 
wind originates from coronal holes, where the density is lower than the 
density in the closed field regions. In order to obtain this observed 
property, we scale the base density, $\rho_0$, and the base temperature, 
$T_0$, at each point on the solar surface with the inverse of the input 
speed from the WSA model. We would like our model to be driven only by the 
magnetogram data without any particular parameterization for each Carrington 
Rotation (CR). Therefore, we parameterize the model for the general cases of 
solar minimum and solar maximum conditions.

The general method to obtain steady-state solution from the Sun to 1AU is to 
run the SC component to steady-state, then turn on IH and couple the two 
components for some time (1 iteration or more). the coupling will drive IH 
through the inner boundary conditions (provided by SC) and the since the 
solar wind is supersonic at this point, a steady-state is obtained quickly 
in IH as well. In principle, it is enough to couple SC-IH for only one 
iteration. It is possible to do the coupling for longer period in order to 
obtain slightly higher magnetic field close to the equator due to the 
different boundary conditions with and without the coupling. Our experience 
however showed that the difference is minimal. 

The SC-IH runs can be done in the HGR coordinate system (the frame rotating 
with the Sun) or in the HGI frame (inertial frame). The HGR run can be done 
in a local time stepping mode, while the HGI run must be done in a 
time-accurate mode. There are sample PARAM.in files to obtain SC-IH 
steady-state solution in either frame:
\begin{verbatim}SWMF_DIR/Param/PARAM.in.test.start.SCIH.HGR\end{verbatim}
\begin{verbatim}SWMF_DIR/Param/PARAM.in.test.start.SCIH.HGI\end{verbatim}
In principle, you only need to change the parameters related to the particular 
CR such as magnetogram file name, \#STARTTIME and satellite files. However, 
from our experience the free parameters of the model should be changed as well.

\section{Model Parameterization \label{section:SCparameterization}}

The SC model was originally planned to have fixed parameters so the only 
change from one CR to another is the input magnetogram. Our experience has 
shown however, that a better solution can be obtained for a particular 
CR by modifying the base density (BodyNDim in the \#BODY command) and the 
magnetogram scaling factor (UnitB in the \#PFSSM command). The value 
range for BodyNDim should be $1\times10^8 - 5\times10^8$ (in $cm^{-3}$) 
and for UnitB should be $1-4$. The recommended scaling factor for MDI 
magnetograms is 1.8 and our experience showed that a value of 2.5 is 
better for solar minimum rotations of 1997. For solar maximum, we recommend 
to use higher value of 4. 

We should note that the fine tuning is important for obtaining good agreement 
with 1AU data, which is very hard when using global model. The 
parameterization should be easier in the case of simulations of the solar 
corona only. A more detailed validation of the model can be found in:
\begin{verbatim}Cohen, O.; Sokolov, I. V.; Roussev, I. I.; Gombosi, T. I., 
Validation of a synoptic solar wind model, 
Journal of Geophysical Research, VOL. 113, A03104, doi:10.1029/2007JA012797, 2008\end{verbatim}

All parameterization above was done using MDI magnetograms. For other data 
sources one should use different values. WSO, MWO, and GONG data seems to have 
weaker field than MDI and SOLIS data. Therefore, a larger scaling factor 
should be used. 
